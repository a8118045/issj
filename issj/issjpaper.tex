%=================================================================
\documentclass[dvipdfmx]{issj}
%=================================================================
% 情報システム学会 研究発表大会 原稿用テンプレートの利用例
%     ISSJ実行委員会/プログラム委員会(ISSJ学会誌編集委員会)
% issj2012.sty Ver. 0.50 2012-10-12 Tadashi Iijima (iijima@ae.keio.ac.jp)
% issj2012.sty Ver. 0.60 2012-10-15 Tadashi Iijima (iijima@ae.keio.ac.jp)
% issj2012.sty Ver. 0.70 2012-10-19 Tadashi Iijima (iijima@ae.keio.ac.jp)
% issj2012.sty Ver. 0.80 2012-10-20 Tadashi Iijima (iijima@ae.keio.ac.jp)
%=================================================================
\title{Twitter上で共感を生み出すツイートの性質に関する考察}
%-----------------------------------------------------------------
\etitle{The template for the proceeding of ISSJ}
%-----------------------------------------------------------------
\author{木村 咲\uddag}
%-----------------------------------------------------------------
\eauthor{Saki Kimura\udag, and Jiro System\uddag}
%-----------------------------------------------------------------
\affiliation{\dag 青山学院大学 社会情報情報学部\ddag }
%-----------------------------------------------------------------
\eaffiliation{\dag Faculty of Information and Communication, System Univ.\\ 
              \ddag Graduate School of Information and Communication, System Univ.}
%-----------------------------------------------------------------
%=================================================================
\begin{document}
%=================================================================
\maketitle
%=================================================================
\begin{abstract}
本研究では、先行研究で実証されていた「Twitter上で共感を生み出すツイートの性質」の再現性を確認するための追試を行った。 追試では先行研究とは異なるツイートデータを対象とし、新たに属性として「画像の有無」と「感情極性値」を追加した。 芸能人4名のデータを収集して分析した結果、全ての属性においてTwitterで共感を生み出す性質は見られなかった。 これは先行研究と(A)ツイートの時期、(B)ラベル付けした被験者が異なるためだと考えられる。
\end{abstract}
%=================================================================
\section{はじめに} %%%% 第1節
%=================================================================

近年はSNS上で買い物ができるようになったり、またYoutuberやインスタグラマーなどSNSを利用して稼ぐ人も増え、今後よりSNSのマーケティングは盛んになるだろう。
中でもSNSを運用するにあたって共感の注目が高まっており、日本企業の約63.9%がSNS運用でユーザーの共感の獲得が目標の指標になっている。[A] 
またSNSで注目されている共感だが、共感については動物行動学、教育心理学、社会心理学、臨床心理学や脳神経生理学など他分野で多面的、学際的にアプローチされ、研究が行われており、その時々で共感の定義そのものも、多様であり、捉え方も様々である。[B]
そこで本稿では、Twitterで共感を生み出すツイートはどのような性質があるのかを調査する。





%========================================================================
\section{先行研究と本研究の目的}  %%%% 第2節
%========================================================================

共感は、従来さまざまな観点から多くの分野で研究されてきた。
大川, 高間の研究[1]では、不特定の相手向けになされたツイート(つぶやき)に対し多数の人が共感を抱くケースに着目し、その発生のメカニズムの解明を行った。
不特定多数のユーザーが閲覧するツイートを対象とするため、ここでは著名人のツイートデータを対象としている。
第一著者の判断により共感が発生しているか否かのラベル付けを行うと同時に、それらのツイートの「いいね数」や「RT数」「文字数」など計11個の属性(表1)の値も取得した。
その結果、「文字数が少ないツイート」と「悲しみを含むツイート」が共感を生み出しやすいことが判明した。
文字数が影響を及ぼしてる理由としては、ツイートの長さが冗長であればあるほどユーザーがテキストの全てを閲覧することがなくなり、共感の発生を妨げる原因になってる可能性があるからだ。
また悲しみを含むツイートは、人間は悲しい記憶が残りやすく、そのためツイートを投稿したユーザーの状況をイメージしやすくなり、共感が発生する可能性が高くなるためだと考察していた。
そこで本研究では、この結果は別のツイートでも再現可能なのかを検証するため、ツイート内容を変えて追試を行った。
加えて本稿ではImage(画像の有無)という新たな属性も加えた。
これは、ツイートに画像が添付されることによってツイート内容のイメージを湧きやすくさせる効果があり、共感にも影響を及ぼすと考えるからだ。



\begin{table}[htbp]\centering
\caption{属性名と属性値の形式}\label{tbl:font}
\begin{small}
\begin{tabular}{|c|c|} \hline
属性名            & 属性値の形式\\\hline\hline
fav & real 型, 0~\\\hline
rt  &  real 型, 0~\\\hline
term  & real 型, 0~\\\hline
characters         & real 型, 0~\\\hline
unofficial      &  real 型, 1~140\\\hline
gladness & {yes, no}\\\hline
anger & {yes, no}\\\hline
sadness & {yes, no}\\\hline
pleasure & {yes, no}\\\hline
sadness & {yes, no}\\\hline
agreement & {agree, disagree, neutral}\\\hline
thinking & {positive, negative, neutral}\\\hline
\end{tabular}
\end{small}
\end{table}



%========================================================================
\section{共感の定義}  %%%% 第3節
%========================================================================

先行研究ではTwitter上におけるツイートに対する共感を「そのツイートを不特定多数のユー ザが閲覧したときに,投稿した背景が想像でき,そ れに同感できる」ことであると定義している。
つまり、ツイートに同感できないと共感は成り立たないことを意味している。
しかし、佐伯は「同感」と「共感」を別物と捉える。
「同感というのはその人の感じていることと自分の感じていることを同じなのだと思うこと」であり,そこには未知なる世界への探求も,新しい発見もなく,「相手は自分と同じだという確認」があるに過ぎない。
一方の共感とは,「白分にはすてきとは思えないが,その良さをわかりたい」というように,「その人が良いといっているのはどういうところなのだろうということを探求して「理解」しようとする。そこにいたる経緯やそこでの状況をしっかり把握して,その場に我が身をおいて,なんとかして,そこでの「良さ」を,心底「納得」しようとする」ことと捉えられている。18).
またロジャーズによると、共感とは「自分が自分が他者であるかのような、しかし”かのようにas if”という状態を失わずに関係する。正確さや感情的要素、意味を持って他者の内部関連気分を知覚すること」と唱えられいてる。
この2つをもとに本研究では共感を、「不特定多数のユーザーがそのツイートを閲覧したとき、投稿した背景が想像でき、そのツイート(投稿者)の内部関連気分を汲み取れること」と定義する。
ゆえに、以下の2項目が成り立つとき、そのツイートは共感できるツイートとする。

\begin{table}[htbp]\centering
\caption{共感できるかの判断基準}\label{tbl:font}
\begin{small}
\begin{tabular}{|c|c|} \hline
項目   & 詳細\\\hline\hline
1& ツイートの背景が記載されているツイート\\\hline
2 & 投稿者の感情が汲み取れるツイート\\\hline
\end{tabular}
\end{small}
\end{table}


%------------------------------------------------------------------------
\subsection{ツイートの収集と共感のラベル付け }  %%%% 第3.1節
%------------------------------------------------------------------------
今回は芸能人の中でもフォロワー数が最も多い芸能人男女4名を分析対象とし、2021年7月27日から最新のツイートデータをそれぞれ50件収集した。
使用するデータは、第一次情報源でない「公式リツイート」と、特定の相手を対象としたつぶやきである「リプライ」は除いている。
TwitterAPIを用いてツイートを取得し、取得したツイートに対して共感したか否かを手作業によってラベル付けを行った。
共感を生み出すとラベル付けされたツイートと、共感を生み出さないとラベル付けされたツイートの例を画像1と画像2に示す。
まず、画像1は共感を生み出すと判断されたツイートだ。
これは、吉高由里子が”自分の誕生日に周りに祝ってもらって嬉しくてツイートした”という背景と、”祝ってもらって嬉しい”という感情が伺える。
これは、図1の2つの条件を満たしているため共感できるツイートだとラベル付けした。
続いて、画像2は共感を生み出さないと判断されたツイートだ。
このツイートからは、吉高由里子が”誰かと手持ち花火をしている”というツイートの背景が確認できる。
しかし、感情面については言及されていないため判断することができない。
つまり、図1の1つの条件を満たしていないため共感を生み出さないツイートとラベル付けした。




%========================================================================
\section{○○}  %%%% 第○節
%========================================================================
%------------------------------------------------------------------------
\subsection{○○}  %%%% 第○○節
%------------------------------------------------------------------------


%========================================================================
%========================================================================
%========================================================================
%========================================================================
%========================================================================
%========================================================================
%========================================================================
%========================================================================
%========================================================================


%========================================================================
\section{原稿の作成}  %%%% 第3節
%========================================================================

\LaTeX 2$\epsilon$を用いる場合,dvipdfmx等によってPDF原稿が作成できます.
角藤版W32{\TeX}\footnote{http://w32tex.org/index-ja.html}で組版が可能となりますよう
御協力をお願いいたします.
Windows OS上でしたら阿部紀行さんのTeXインストーラ3\footnote{%
http://www.math.sci.hokudai.ac.jp/~abenori/soft/abtexinst.html}で
\LaTeX 2$\epsilon$環境をインストールされることをお奨めします
(動作確認もその環境で行っています).

なお,PDF作成時にセキュリティロックをかけないようお願い致します
(標準の設定ではロックをかけないように設定されています).

%========================================================================
\section{原稿の提出}  %%%% 第4節
%========================================================================

組版したPDF原稿と{\LaTeX}原稿(原稿とともに図版ファイルや追加スタイルファイルなど
組版に必要なファイルのすべてをアーカイブしたzipファイル)の両方を
提出していただくことが必要です.
前述の推奨環境以外の\LaTeX 2$\epsilon$環境をご使用を希望される場合,ならびに,
角藤版W32TeXで組版できるかどうか明確でない場合,スタイルファイルに不具合がある場合には,
\begin{center}
    issj-office@issj.net
\end{center}
までご相談ください.

 予稿論文原稿は,組版したPDF原稿とTeX原稿の両方を,
御提出いただきます.

具体的な方法は,大会WWWページ
\begin{center}
    http://www.issj.net/conf/
\end{center}
に掲示いたしますので,よろしくお願いいたします.


%========================================================================
\section{本文の書き方}  %%%% 第5節
%========================================================================

本文は原則として1段組(目安は1行あたり46字)でお書きください.
タイトルから本文までを含め,全体で2頁から4頁までとします.
ただしロングについては,2頁から6頁とします.

%------------------------------------------------------------------------
\subsection{フォントについて}  %%%% 第5.1節
%------------------------------------------------------------------------

予稿集に用いるフォントは原則としてこのスタイルに従ってください.

\vskip \baselineskip

\noindent~フォントの種類:
このスタイルでは,和文は見出しのみゴシック(太字),その他は明朝体,英文は基本的に{\LaTeX}で標準的な
Computer Modernとなっています.

\vskip \baselineskip

\noindent~フォントの大きさ:
\begin{quotation}
\begin{tabular}{ll}
タイトル & 16ポイント\\
著者名   & 11ポイント\\
要旨     & 10ポイント\\
本文     & 11ポイント
\end{tabular}
\end{quotation}

%------------------------------------------------------------------------
\subsection{図表について}  %%%% 第5.2節
%------------------------------------------------------------------------

図表の番号は,次頁の例を参考に図\ref{fig:example}, 表\ref{tbl:font}などとしてください.
図はカラーでも結構です.
原則として,図キャプションは図の下,表キャプションは表の上に表示してください.

\vskip \baselineskip

\noindent~図表の作成例:

\begin{figure}[htbp]\centering
\includegraphics{fig1.png}
\caption{図の例}\label{fig:example} 
\end{figure}

\begin{table}[htbp]\centering
\caption{本テンプレートにおけるフォントのサイズ}\label{tbl:font}
\begin{small}
\begin{tabular}{|c|c|} \hline
Part             & Font size (point)\\\hline\hline
Title (Japanese) & 16\\\hline
Title (English)  & 14.4\\\hline
Author           & 11\\\hline
Abstract         & 10\\\hline
Body             & 11\\\hline
\end{tabular}
\end{small}
\end{table}

%------------------------------------------------------------------------
\subsection{数式について}  %%%% 第5.3節
%------------------------------------------------------------------------

数式は原則として,{\LaTeX}の機能を使って組版してください.
次にサンプルを示します.

\begin{equation}
	p( \lambda \vert y  ) = \frac{p( y \vert \lambda ) p( \lambda )}{p( y )}
\end{equation}


数式にはこの例のように右隅に参照用の番号をつけてください.

%------------------------------------------------------------------------
\subsection{参考文献について}  %%%% 第5.4節
%------------------------------------------------------------------------

参考文献は本文中で引用された順に採番し,角カッコ付きで\cite{bib:SO2004},\cite{bib:HV1987},
\cite{bib:NT1996}などと表示してください.

\begin{itemize}
\item[] 雑誌は,本テンプレートの例の\cite{bib:SO2004},\cite{bib:HV1987}に従ってください.
\item[] 著書は,本テンプレートの例の\cite{bib:NT1996},\cite{bib:K19751976}に従って,和・英文ともに,
      \begin{center}
          著書名, 書名, 発行所名, 発行年(西暦)[, 頁]
      \end{center}の順に記載してください.
\end{itemize}

%========================================================================
\section{まとめ}  %%%% 第6節
%========================================================================

以上,本テンプレートにしたがって原稿作成をお願いいたします.不明な点については,
\begin{center}
    issj-office@issj.net
\end{center}
まで,お問い合わせください.

%========================================================================
%%%% 参考文献
%========================================================================
\begin{thebibliography}{99}
  \bibitem{bib:SO2004} 斎藤一, 大内東, 
                       ^^ ^^ 組織評価における能力成熟度モデルの適用 -- 観光関係部局の調査結果について, '' 
                       情報処理学会論文誌, Vol.45 No.3, 2004, pp.809-812.
  \bibitem{bib:HV1987} Harker, P.T. and Vargas, L.G., 
                       ^^ ^^ The Theory of Ratio Scale Estimation: Saaty’s Analytic Hierarchy Process, '' 
                       Management Science, Vol.33, 1987, pp.1383-1403.
  \bibitem{bib:NT1996} 野中郁次郎, 竹内弘高, ^^ ^^ 知識創造企業, '' 東洋経済新報社, 1996.
  \bibitem{bib:K19751976}  Kleinrock L., 
                       ^^ ^^ Queuing Systems, '' Volume 1, 2, John Wiley \& Sons, Inc., 1975, 1976.
\end{thebibliography}
%========================================================================
\end{document}
%========================================================================
