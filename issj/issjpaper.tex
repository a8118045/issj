%=================================================================
\documentclass[dvipdfmx]{issj}
%=================================================================
% 情報システム学会 研究発表大会 原稿用テンプレートの利用例
%     ISSJ実行委員会/プログラム委員会(ISSJ学会誌編集委員会)
% issj2012.sty Ver. 0.50 2012-10-12 Tadashi Iijima (iijima@ae.keio.ac.jp)
% issj2012.sty Ver. 0.60 2012-10-15 Tadashi Iijima (iijima@ae.keio.ac.jp)
% issj2012.sty Ver. 0.70 2012-10-19 Tadashi Iijima (iijima@ae.keio.ac.jp)
% issj2012.sty Ver. 0.80 2012-10-20 Tadashi Iijima (iijima@ae.keio.ac.jp)
%=================================================================
\title{Twitter上で共感を生み出すツイートの性質に関する考察}
%-----------------------------------------------------------------
\etitle{The template for the proceeding of ISSJ}
%-----------------------------------------------------------------
\author{木村 咲\uddag}
%-----------------------------------------------------------------
\eauthor{Saki Kimura\udag, and Jiro System\uddag}
%-----------------------------------------------------------------
\affiliation{\dag システム大学 情報学部\\ 
             \ddag システム大学大学院 情報学研究科}
%-----------------------------------------------------------------
\eaffiliation{\dag Faculty of Information and Communication, System Univ.\\ 
              \ddag Graduate School of Information and Communication, System Univ.}
%-----------------------------------------------------------------
%=================================================================
\begin{document}
%=================================================================
\maketitle
%=================================================================
\begin{abstract}
この部分に和文要旨を書いてください(目安は\textcolor{red}{300字}程度)
\end{abstract}
%=================================================================
\section{はじめに} %%%% 第1節
%=================================================================

この文書は,\LaTeX 2$\epsilon$用の簡単な利用例になっています.
この書式にしたがって発表予稿論文の原稿を御作成ください.また,
この文書のページ設定や書式は変更しないでください.

投稿者からの要望によって,スタイルファイルは,適宜,改善・改良されることがあります.
随時,大会WWWページをご確認ください.

発表者は,大会WWWページの指示に基づいて,期日までに,組版したPDF原稿と{\LaTeX}原稿
(本文テキストファイルと図版ファイル,および必要な追加スタイルファイルがあるならそれを
zip形式でアーカイブしたファイル)の両方を提出してください
(PDF原稿はWWWページへアップロードしてください.{\LaTeX}原稿の提出方法はWWWページの
指示に従ってください).
プログラム委員会(大会事務局)にて編集した上で電子ジャーナル形式の予稿集として
本学会Webサイトに掲載いたします(必要に応じて,提出された{\LaTeX}原稿から
組版しなおすこともあります).
また,投稿されても,研究発表大会で発表されなかった論文は
最終的に予稿集から削除されますので,
御承知おきくださいますようお願いします.

%========================================================================
\section{スタイルファイルについて}  %%%% 第2節
%========================================================================

このスタイルでは,用紙サイズ(A4縦置き),余白(上下左右の余白すべて20mm)が
適切に設定されています.

タイトル,著者,所属は,レイアウト枠内に記載してください
(ただし,英文の著者と所属の記載は省略可能です).
本文は,1段組の設定がなされています.

なお,各ページのヘッダとフッタ,ならびにページ番号は,
予稿集編集時にプログラム委員会(大会事務局)にて設定致しますので,
変更しないでください.


%========================================================================
\section{原稿の作成}  %%%% 第3節
%========================================================================

\LaTeX 2$\epsilon$を用いる場合,dvipdfmx等によってPDF原稿が作成できます.
角藤版W32{\TeX}\footnote{http://w32tex.org/index-ja.html}で組版が可能となりますよう
御協力をお願いいたします.
Windows OS上でしたら阿部紀行さんのTeXインストーラ3\footnote{%
http://www.math.sci.hokudai.ac.jp/~abenori/soft/abtexinst.html}で
\LaTeX 2$\epsilon$環境をインストールされることをお奨めします
(動作確認もその環境で行っています).

なお,PDF作成時にセキュリティロックをかけないようお願い致します
(標準の設定ではロックをかけないように設定されています).

%========================================================================
\section{原稿の提出}  %%%% 第4節
%========================================================================

組版したPDF原稿と{\LaTeX}原稿(原稿とともに図版ファイルや追加スタイルファイルなど
組版に必要なファイルのすべてをアーカイブしたzipファイル)の両方を
提出していただくことが必要です.
前述の推奨環境以外の\LaTeX 2$\epsilon$環境をご使用を希望される場合,ならびに,
角藤版W32TeXで組版できるかどうか明確でない場合,スタイルファイルに不具合がある場合には,
\begin{center}
    issj-office@issj.net
\end{center}
までご相談ください.

 予稿論文原稿は,組版したPDF原稿とTeX原稿の両方を,
御提出いただきます.

具体的な方法は,大会WWWページ
\begin{center}
    http://www.issj.net/conf/
\end{center}
に掲示いたしますので,よろしくお願いいたします.


%========================================================================
\section{本文の書き方}  %%%% 第5節
%========================================================================

本文は原則として1段組(目安は1行あたり46字)でお書きください.
タイトルから本文までを含め,全体で2頁から4頁までとします.
ただしロングについては,2頁から6頁とします.

%------------------------------------------------------------------------
\subsection{フォントについて}  %%%% 第5.1節
%------------------------------------------------------------------------

予稿集に用いるフォントは原則としてこのスタイルに従ってください.

\vskip \baselineskip

\noindent~フォントの種類:
このスタイルでは,和文は見出しのみゴシック(太字),その他は明朝体,英文は基本的に{\LaTeX}で標準的な
Computer Modernとなっています.

\vskip \baselineskip

\noindent~フォントの大きさ:
\begin{quotation}
\begin{tabular}{ll}
タイトル & 16ポイント\\
著者名   & 11ポイント\\
要旨     & 10ポイント\\
本文     & 11ポイント
\end{tabular}
\end{quotation}

%------------------------------------------------------------------------
\subsection{図表について}  %%%% 第5.2節
%------------------------------------------------------------------------

図表の番号は,次頁の例を参考に図\ref{fig:example}, 表\ref{tbl:font}などとしてください.
図はカラーでも結構です.
原則として,図キャプションは図の下,表キャプションは表の上に表示してください.

\vskip \baselineskip

\noindent~図表の作成例:

\begin{figure}[htbp]\centering
\includegraphics{fig1.png}
\caption{図の例}\label{fig:example} 
\end{figure}

\begin{table}[htbp]\centering
\caption{本テンプレートにおけるフォントのサイズ}\label{tbl:font}
\begin{small}
\begin{tabular}{|c|c|} \hline
Part             & Font size (point)\\\hline\hline
Title (Japanese) & 16\\\hline
Title (English)  & 14.4\\\hline
Author           & 11\\\hline
Abstract         & 10\\\hline
Body             & 11\\\hline
\end{tabular}
\end{small}
\end{table}

%------------------------------------------------------------------------
\subsection{数式について}  %%%% 第5.3節
%------------------------------------------------------------------------

数式は原則として,{\LaTeX}の機能を使って組版してください.
次にサンプルを示します.

\begin{equation}
	p( \lambda \vert y  ) = \frac{p( y \vert \lambda ) p( \lambda )}{p( y )}
\end{equation}


数式にはこの例のように右隅に参照用の番号をつけてください.

%------------------------------------------------------------------------
\subsection{参考文献について}  %%%% 第5.4節
%------------------------------------------------------------------------

参考文献は本文中で引用された順に採番し,角カッコ付きで\cite{bib:SO2004},\cite{bib:HV1987},
\cite{bib:NT1996}などと表示してください.

\begin{itemize}
\item[] 雑誌は,本テンプレートの例の\cite{bib:SO2004},\cite{bib:HV1987}に従ってください.
\item[] 著書は,本テンプレートの例の\cite{bib:NT1996},\cite{bib:K19751976}に従って,和・英文ともに,
      \begin{center}
          著書名, 書名, 発行所名, 発行年(西暦)[, 頁]
      \end{center}の順に記載してください.
\end{itemize}

%========================================================================
\section{まとめ}  %%%% 第6節
%========================================================================

以上,本テンプレートにしたがって原稿作成をお願いいたします.不明な点については,
\begin{center}
    issj-office@issj.net
\end{center}
まで,お問い合わせください.

%========================================================================
%%%% 参考文献
%========================================================================
\begin{thebibliography}{99}
  \bibitem{bib:SO2004} 斎藤一, 大内東, 
                       ^^ ^^ 組織評価における能力成熟度モデルの適用 -- 観光関係部局の調査結果について, '' 
                       情報処理学会論文誌, Vol.45 No.3, 2004, pp.809-812.
  \bibitem{bib:HV1987} Harker, P.T. and Vargas, L.G., 
                       ^^ ^^ The Theory of Ratio Scale Estimation: Saaty’s Analytic Hierarchy Process, '' 
                       Management Science, Vol.33, 1987, pp.1383-1403.
  \bibitem{bib:NT1996} 野中郁次郎, 竹内弘高, ^^ ^^ 知識創造企業, '' 東洋経済新報社, 1996.
  \bibitem{bib:K19751976}  Kleinrock L., 
                       ^^ ^^ Queuing Systems, '' Volume 1, 2, John Wiley \& Sons, Inc., 1975, 1976.
\end{thebibliography}
%========================================================================
\end{document}
%========================================================================
